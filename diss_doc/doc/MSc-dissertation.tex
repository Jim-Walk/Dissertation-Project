\documentclass[12pt,a4paper]{report}

\usepackage{epcc}
\usepackage{graphicx}
\usepackage{alltt}
\usepackage{listings, listings-rust}
\lstset{
  basicstyle=\footnotesize,
  captionpos=b
}
\usepackage{minted}
\usepackage{todonotes}
\usepackage{tikz}
\usepackage{float}

\begin{document}
\setminted{
    fontsize=\scriptsize,
    frame=single}

%AC%\pagestyle{myheadings}
%AC%\markright{D.~S.~Henty}

%\title{A Latex thesis example}
%\author{D.~S.~Henty}
%\date{\today}

%\maketitle

\pagenumbering{roman}

\title{High Performance Rust}
\author{Jim Walker}
\date{\today}

\makeEPCCtitle

\thispagestyle{empty}

\vspace{12cm}

\begin{center}

\large{MSc in High Performance Computing}

\large{The University of Edinburgh}

\large{Year of Presentation: 2019}

\end{center}

\newpage

\begin{abstract}
This dissertation examines the suitabilty of the Rust programming language, to High Performance Computing (HPC). This examination is made through porting three HPC mini apps to Rust from typical HPC languages and comparing the perfomance of the Rust and the original implementation. We also investigate the readability of Rust's higher level programming syntax for HPC programmers through the use of a questionnaire.
\end{abstract}

\pagenumbering{roman}

\tableofcontents
\listoftables
\listoffigures
\lstlistoflistings

\begin{titlepage}
\vspace*{2in}
% an acknowledgements section is completely optional but if you decide
% not to include it you should still include an empty {titlepage}
% environment as this initialises things like section and page numbering.
\section*{Acknowledgements}

This template is a slightly modified version of the one developed by
Prof. Charles Duncan for MSc students in the Dept. of Meteorology. His
acknowledgement follows:

{\em This template has been produced with help from many former students who
have shown different ways of doing things. Please make suggestions for
further improvements.}

\end{titlepage}

\pagenumbering{arabic}

\chapter{Introduction}
The Rust programming language promises 'High-level ergonomics and low-level control' to help 'you write faster, more reliable software' \cite{RustBook}

\section{Mini App Selection}


\chapter{Background}
\section{High Performance Computing}
What even is it lol

\section{Kernels}
By Kernels I mean blah blah. I will use Kernels in a similar way to how Mini-apps have been used in research in the past.

Mini-apps are a well established method of assessing new programming languages or techniques within HPC~\cite{Mallinson:2014, Slaughter:2015, martineau2017arch}. A mini-app is a small program which reproduces some functionality of a common HPC use case. Often, the program will be implemented using one particular technology, and then ported to another technology. The performance of the two mini-apps will then be tested, to see which technology is better suited to the particular problem represented by that mini-app. Such an approach gives quantitative data which provides a strong indication for the performance of a technology in a full implementation of an application. I am going to use Kernels rather than mini-apps because more bredth and less time, more use cases, better indication

This dissertation will follow a similar approach of evaluating a program through the performance of a kernel, using the test data to find any weaknesses in the Rust or original implementation.

I will also evaluate the ease with which I am able to port a kernel into Rust. These observations will provide insight into what it is like to program in Rust, if its strict memory model and functional idioms help or hinder translation from the imperative languages which the ported programs are written in. This qualitative, partly experiential information will hopefully provide an insight into the actual practicalities of programming in Rust. For Rust to be fully accepted by the HPC community, it is necessary that the program fulfils the functional requirements of speed and scaling, alongside non functional requirements, of usability and user experience. The first factor provides a reason for using Rust programs in HPC, the second provides an impetus for learning how to write those programs


\todo{Why use reference implementations and not write my own?}

\section{C/C++}
When was it developed, who by etc etc, how is it used in HPC today? Which compiler am I using? Common memory safety issues of C, how C++ tries to fix them
\subsection{OpenMP}

\section{Rust}
Who developed it? Why?

mention borrow checker. Which rustc version am I using? rustc 1.34.2
\subsection{Rayon}
Talk about the underlying nature of Rayon and its random scheduling. Not official library for easy parallelism but it's used a lot in the book.


\chapter{Methodology}
\section{Kernel Selection}
So that a breadth of usage scenarios were examined, three kernels were selected based on their conformity to the following set of criteria.
\begin{itemize}
  \item \textbf{The part of the program responsible for more than two thirds of the processing time should not be more than 1500 lines.} To ensure that I fully implemented three ports of existing kernels, it was necessary to limit the size of the kernels that could be considered. This was an unfortunately necessary decision to make. Whilst it reduced the field of possible kernels, an analysis of the rejected kernels found that many of them devoted lots of code to subtle computational variations, which were of more importance to a particular rarefied domain, rather than presenting a novel approach to parallelism. (i could cite some benchmarks here like bookleaf or something)

  \item \textbf{The program must use shared memory parallelism and target the CPU.} Rust's (supposed) zero cost memory safety features are its differentiating factor. The best way to test the true cost of Rust's memory safety features would be through shared memory parallelism, where a poor implementation of memory management will make itself evident through poor performance. Programs which target the GPU rather than the CPU will not be considered, as the current implementations for Rust to target GPUs involve calling out to existing GPU APIs. Therefore, any analysis of a Rust program targeting a GPU would largely be an analysis of the GPU API itself.

  \item \textbf{The program run time should reasonably decrease as the number of threads increases, at least until the number of threads reaches 32.} It is important that any kernel considered is capable of scaling to the high core counts normally seen in HPC.I will be running the kernels on Cirrus, which supports 36 real threads.

  \item \textbf{The program operate on data greater than the CPU's L3 Cache} so that we can be sure that the kernel is representative of working on large data sets. Cirrus has an L3 cache of 45MiB. As each node has 256GB of RAM, a central constraint when working with large data sets is the speed with which data is loaded into the cache. Speed is often achieved by programs in this area through vectorisation, the use of which can be deduced from a program's assembly code. If there is a large performance difference between Rust and the reference kernels, we can use the program's assembly code to reason about that difference.

  \item \textbf{The program must be written in C or C++.} This restriction allows us to choose work which is more representative of HPC programs that actually run on HPC systems, rather than python programs which call out to pre-compiled libraries. Unlike Fortran, C and C++ use array indexing and layout conventions similar to Rust, which will make porting programs from them easier.

  \item \textbf{The program must use OMP.} This is a typical approach for shared memory parallelism in HPC. Use of a library to do the parallel processing also further standardises the candidate programs, which will lead to a deeper understanding of the kernel's performance factors.
\end{itemize}

The best kernels I found to fit this criteria were Babel Stream, sparse matrix vector multiplication and K-means clustering.

Babel Stream was developed by blah blah and it is written in C++, using OpenMP it has a good precedent of being used by blah blah, it performs operations blah blah. Tests found it to scale well so I implemented it.

\subsection{Implementation}
Implementation of all three programs follows the same process, as outlined in Figure~\ref{fig:imp-flow}. Once a candidate kernel is selected, it is implemented in Rust in serial. Any differences between the  behaviour of the Rust and the original implementation are thought of as bugs, and are eradicated or minimised as far as is possible. For ease of development, the Rust crate Clap was used to read command line arguments for the program, leading to Rust implementations of kernels being called with slightly different syntax. This difference was deemed to be superficial enough to be allowable. Kernel output was ensured to be as similar as possible to aid data-collection from both implementations.

\begin{figure}
  \center
  \includegraphics[height=12cm]{figs/ImplementationFlow.png}
  \caption{Flow Diagram for Implementation Process}
  \label{fig:imp-flow}
\end{figure}

Next, I would eliminate any bugs found in my serial implementation of the code by comparing outputs between my implementation and the reference implementation. During this process I would also move the code away from its C conventions towards more idiomatic Rust. To achieve more idiomatic Rust, I used the linting tool Clippy~\cite{RustClippy}, which was developed by the Rust team.  Clippy includes a category of lints under  which highlight `code that should be written in a more idiomatic way'~\cite{RustClippy}. I implemented all of Clippy's recommended rewrites, which would often include replacing the use of for loops to access vector variables with calls to the vectors \texttt{iter()} method. This particular replacement could require code to be rewritten in a much more functional style. (should I give an example?)

I would then parallelise the kernel using Rayon~\cite{RustRayon} at the same loops where the reference implementation used OpenMP to parallelise its loops. Sometimes this would be a simple matter of replacing the \texttt{iter()} method with \texttt{par_iter()}, but more parallising more complex operations like reductions and initilisations was slightly more difficult.

I would then again endeavour to fix bugs Bugs at this stage could be hard to fix as they could come from original implementations

testing is discussed in full detail next section, box could instead say `Ready for testing'

\subsubsection{Babel Stream}
\begin{itemize}
  \item Type problems due to generics leading to verbose code and obfuscating debugging
  \item The compiler did help with type debugging a little, but had limitations - give example
  \item Idiomatic serial Rust was faster than C like rust, potentially due to iter\_mut allowing optimisations? Evidence from triad and add.
  \item Once I figured out the for\_each pattern is was easy to apply it to other operations
  \item Realised that Rust's serial init was a bottleneck
  \item difficulty in writing para init as not a common use case scenario, and obfusticated by type
\end{itemize}

Initialisation is the very verbose - Explain why it's so verbose, process for finding this to be worth doing etc.
\begin{lstlisting}[language=Rust]
vec![0.0; arr_size].par_iter()
                   .map(|_| T::from(0.2).unwrap())
                   .collect_into_vec(&mut self.b);
\end{lstlisting}

Explain what's going on in this code fragment, compare it to the C original. Whilst this is a lot, Rust does reduce need for defensive coding. Could do a sloc comparison between original and new, if it was felt to be worth doing.

\begin{lstlisting}[language=Rust]
pub fn triad(&mut self){
  let scalar_imut = self.scalar;
  self.a.par_chunks_mut(self.chunk_size)
        .zip(self.c.par_chunks(self.chunk_size))
        .zip(self.b.par_chunks(self.chunk_size))
        .for_each(|((a, c), b)|
              for ((a_i, c_i,), b_i) in a.iter_mut()
                                         .zip(c.iter())
                                         .zip(b.iter()){
                                           *a_i = *b_i + scalar_imut * *c_i
                                          });
}
\end{lstlisting}

\subsubsection{Sparse Matrix}
\begin{itemize}
  \item Bit shift overflow causes Rust to crash not just run on, have to be more careful about kernel input parameters. Initially thought this might be a bug. Give example.
  \item Found init bug, was very difficult to implement para init. Filed bug report with original project
  \item remember that class you tried to build to increment stuff? m8
\end{itemize}

\subsection{Experimentation}

\section{Questionnaire}


\chapter{Results}
\section{Babel Stream}
\begin{figure}[h]
  \center
  \includegraphics[width=.8\linewidth]{figs/babel/Dot.png}
  \caption{Babel Stream: Dot product bandwidth}
  \label{fig:babel-dot}
\end{figure}
Babel stream results show that Rayon is unable to scale as well as OpenMP. This is likely because of the CC-Numa layout of the system, and Rayon's affinity schedule. I might include a diagram here to show exactly what I mean. Is it worth getting the legend outside the box? would make things look neater

I might generate this figure again but without the 500MB rust run, and include chunkless para init runs instead on add, triad etc
\section{Sparse Matrix}
\section{K-means}
\section{Questionnaire}



% in practice you would probably keep this in a separate file and use
% the \include{filename} command to insert it here.



\chapter{Conclusions}

This is the place to put your conclusions about your work. You can
split it into different sections if appropriate. You may want to include
a section of future work which could be carried out to continue your
research.

\appendix
% the appendix command just changes heading styles for appendices.

\chapter{Stuff which is too detailed}

Appendices should contain all the material which is considered too
detailed to be included in the main bod but which is, nevertheless,
important enough to be included in the thesis.

\chapter{Stuff which no-one will read}

Some people include in their thesis a lot of detail, particularly
computer code, which no-one will ever read. You should be careful that
anything like this you include should contain some element of uniqueness
which justifies its inclusion.

\bibliographystyle{plain}
\bibliography{bib}

\end{document}
