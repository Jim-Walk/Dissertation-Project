\chapter{Background}
\section{Kernels}
By Kernels I mean blah blah. I will use Kernels in a similar way to how Mini-apps have been used in research in the past.

Mini-apps are a well established method of assessing new programming languages or techniques within HPC~\cite{Mallinson:2014, Slaughter:2015, martineau2017arch}. A mini-app is a small program which reproduces some functionality of a common HPC use case. Often, the program will be implemented using one particular technology, and then ported to another technology. The performance of the two mini-apps will then be tested, to see which technology is better suited to the particular problem represented by that mini-app. Such an approach gives quantitative data which provides a strong indication for the performance of a technology in a full implementation of an application. I am going to use Kernels rather than mini-apps because more bredth and less time, more use cases, better indication

This dissertation will follow a similar approach of evaluating a program through the performance of a kernel, using the test data to find any weaknesses in the Rust or original implementation.

I will also evaluate the ease with which I am able to port a kernel into Rust. These observations will provide insight into what it is like to program in Rust, if its strict memory model and functional idioms help or hinder translation from the imperative languages which the ported programs are written in. This qualitative, partly experiential information will hopefully provide an insight into the actual practicalities of programming in Rust. For Rust to be fully accepted by the HPC community, it is necessary that the program fulfils the functional requirements of speed and scaling, alongside non functional requirements, of usability and user experience. The first factor provides a reason for using Rust programs in HPC, the second provides an impetus for learning how to write those programs

\section{C/C++}
When was it developed, who by etc etc, how is it used in HPC today? Which compiler am I using? Common memory safety issues of C, how C++ tries to fix them
\subsection{OpenMP}

\section{Rust}
Who developed it? Why?

mention borrow checker. Which rustc version am I using? rustc 1.34.2
\subsection{Rayon}
Talk about the underlying nature of Rayon and its random scheduling. Not official library for easy parallelism but it's used a lot in the book.
