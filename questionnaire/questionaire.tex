\documentclass[11pt]{article} % Default font size is 12pt, it can be changed here

\usepackage{geometry} % Required to change the page size to A4
\usepackage{listings} % for code
\usepackage{tasks}
\usepackage{exsheets}
\SetupExSheets[question]{type=exam}
\geometry{a4paper} % Set the page size to be A4 as opposed to the
\lstset{basicstyle=\footnotesize}
\begin{document}
\author{Jim Walker}
\title{High Performance Rust Questionnaire}
\maketitle
Hi! I'm Jim Walker, one of the MSc students at EPCC. My dissertation aims to examine if the new programming language, Rust, is suitable for HPC. To thatt end, I have written this questionnaire, to asses how easy it is for HPC programmers such as yourself to understand it.

The questionnaire is simple. There are five questions, which present you with a fragment of rust code. Please describe what each of these functions does to the best of your abilities
\newsavebox\myboxa
\begin{lrbox}{\myboxa}
  \begin{minipage}{\textwidth}
    \begin{lstlisting}
let mut v1 = vec![2,8];
let v2 = vec![2;8];
\end{lstlisting}
\end{minipage}
\end{lrbox}

\begin{question}

\noindent\usebox\myboxa

\begin{tasks}(4)
  \task Assigns label v1 to a mutable vector of elements 2 and 8. Assigns label v2 to a vector with 8 elements of value 2.
  \task Assigns label v1 to a mutated vector with elements 2 and 8. Assigns label v2 to a vector of 2 elements, both with value 8.
  \task Assigns label v1 to a mutatable vector with elements 2 and 8. Assigns label v2 to a vector of 2 elements, both with value 8.
  \task Assigns label v1 to a mutable vector of 8 elements of value 2. Assigns label v2 to a vector with elements 2 and 8.
\end{tasks}

\end{question}

\newsavebox\myboxb
\begin{lrbox}{\myboxb}
  \begin{minipage}{\textwidth}
    \begin{lstlisting}
v.iter().fold(1, |foo, x| foo * x);
\end{lstlisting}
\end{minipage}
\end{lrbox}

\begin{question}
In this question, please assume that v is a vector.

\noindent\usebox\myboxb

\begin{tasks}(4)
  \task An iteretator is created over the vector v, which calls the annonymous/lambda function on each element of the vector.  
  \task Every element of v is multiplied together
  \task thing3
  \task thing4
\end{tasks}

\end{question}

\newsavebox\myboxc
\begin{lrbox}{\myboxc}
  \begin{minipage}{\textwidth}
    \begin{lstlisting}
pub fn foo(&mut self)->T {

    self.a.par_chunks(self.chunk_size)
        .zip(self.b.par_chunks(self.chunk_size))
        .map(|(a,b)| a.iter()
                      .zip(b.iter())
                      .fold(T::from(0).unwrap(), | acc, ele| acc + *ele.0 * *ele.1)
            )
        .sum()
}
\end{lstlisting}
\end{minipage}
\end{lrbox}

\begin{question}

\noindent\usebox\myboxc

\begin{tasks}(4)
  \task Dot product
  \task thing2
  \task thing3
  \task thing4
\end{tasks}

\end{question}

\end{document}
