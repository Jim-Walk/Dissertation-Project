\documentclass[11pt]{article} % Default font size is 12pt, it can be changed here

\usepackage{geometry} % Required to change the page size to A4
\usepackage{listings} % for code
\geometry{a4paper}
\usepackage[clear-aux]{xsim}

\usepackage{enumitem,amssymb,fmtcount}
\newlist{choices}{itemize}{1}
\setlist[choices]{label=$\Box$}
\newcommand*\choice{\item}

\DeclareExerciseProperty{choices}
\DeclareExerciseProperty*{multiple}
\DeclareExerciseEnvironmentTemplate{mc}
  {%
    \UseExerciseTemplate{begin}{default}%
    \IfExerciseBooleanPropertyTF{multiple}
      {}
      {%
        \GetExercisePropertyT{choices}
          {}%
      }%
    \begin{choices}
  }
  {%
    \end{choices}
    \UseExerciseTemplate{end}{default}%
  }

\DeclareExerciseType{mc}{
  exercise-env = multiplechoice ,
  solution-env = correctchoices ,
  exercise-name = Question ,
  solution-name = Solution ,
  exercise-template = mc ,
  solution-template = mc ,
  counter = exercise
}

\xsimsetup{
  exercise/name = Question ,
  solution/name = Solution
}

\lstset{basicstyle=\footnotesize}
\begin{document}
\author{Jim Walker}
\begin{center}
\large {High Performance Rust MSc Questionnaire}

\normalsize Jim Walker

s1893750@ed.ac.uk
\end{center}

Hi! I'm Jim Walker, one of the MSc students at EPCC. My dissertation aims to examine if the new programming language Rust, is suitable for HPC. To that end, I have written this questionnaire, to assess how easy it is for HPC programmers, such as yourself to understand it.

All data collected through this questionnaire will be annonymous. It will only be used for this dissertation, and will be destroyed once the dissertation has been graded.

\begin{multiplechoice}[choices=1]
What does the function \texttt{foo} do?
\begin{lstlisting}
fn foo(m: i32, n: i32) -> i32 {
   if m == 0 {
      n.abs()
   } else {
      foo(n % m, m)
      test()
   }
}
\end{lstlisting}
  \choice It finds the greatest common divisor of m and n
  \choice It doesn't compile.
  \choice It finds the closest prime number to n
  \choice It calls itself infinitely.
\end{multiplechoice}

\begin{multiplechoice}[choices=1]
In Rust, \texttt{vec!} is used to create a vector. All variables in Rust are immutable by default. What happens when we try to run this program?
\begin{lstlisting}
let v = vec![2,3];
v.push(3);
print!("{:?}", v);
\end{lstlisting}

\choice {[2,3,2]} is printed.
\choice {[2,2,2,3]} is printed.
\choice The program does not compile.
\choice The program compiles, but crashes when it tries to push 3 to v.

\end{multiplechoice}


\begin{multiplechoice}[choices=1]
Idomatic Rust code oten uses patterns associated with functional languages. Given an immutable vector, v, please select what the line of code below does.
\begin{lstlisting}
let a = v.iter().fold(1, |acc, x| acc * x);
\end{lstlisting}
  \choice Every element of v is set to 1, and then copied to a.
  \choice Every element of v is multiplied together and the result is stored in a.
  \choice Every element of v is multiplied by 1 and the result is stored in a.
  \choice The program does not compile.
\end{multiplechoice}

\begin{multiplechoice}[choices=1]
A vector's push method return an optional value, or none. What does this fragment of code print?
\begin{lstlisting}
let mut stack = Vec::new();

stack.push(1);
stack.push(2);
stack.push(3);

while let Some(top) = stack.pop() {
  print!("{} ", top);
}
\end{lstlisting}
  \choice Some(3) Some(2) Some(1)
  \choice 3 2 1 None None None...
  \choice 3 2 1
  \choice Some(3) Some(2) Some(1) None None None...
\end{multiplechoice}


\begin{multiplechoice}[choices=1]
What is a set to?

\begin{lstlisting}
let a: Vec<i32> = (1..).step_by(3)
                       .take(3)
                       .map(|x| x * 2)
                       .collect();
\end{lstlisting}

  \choice {[2, 4, 6]}
  \choice The program doesn't compile.
  \choice {[4, 10, 16]}
  \choice {[2. 8, 14]}
\end{multiplechoice}

\begin{multiplechoice}[choices=1]
In this question, a and b are both vectors of the same length. The method \texttt{par\_chunks} returns a parallel iterator over at most \texttt{chunk\_size} elements at a time. What does this fragment of code do?
\begin{lstlisting}
a.par_chunks(chunk_size)
    .zip(b.par_chunks(chunk_size))
    .map(|(x,y)| x.iter()
                  .zip(y.iter())
                  .fold(0, | acc, ele| acc + *ele.0 * *ele.1)
    ).sum();

\end{lstlisting}

\choice Sum reduction
\choice Dot Product
\choice Transpose
\choice A single iteration of a one dimensional relaxation.

\end{multiplechoice}
\begin{multiplechoice}[choices=1]
The Rust compiler's borrow checker makes sure that values are mutably borrowed if they are altered from a different function than the one they were created in. What does this program print?
\begin{lstlisting}
fn plus_one(x: &mut i32){
    *x += 1;
}
fn main(){
    let x = 64;
    plus_one(&mut x);
    println!("{}", x+1);
}
\end{lstlisting}
  \choice 65
  \choice Undefined.
  \choice It doesn't compile.
  \choice 66.
\end{multiplechoice}
\pagebreak
\begin{exercise}
Please tick the boxes below to show your level of skill in the varying programming languages
\begin{center}
\begin{tabular}{|l|c|c|c|c|}
\hline
& None & Basic & Comprehensive & Advanced \\ \hline
Fortran & & & & \\ \hline
C & & & & \\ \hline
C++ & & & & \\ \hline
Python & & & & \\ \hline
Ruby & & & & \\ \hline
Java & & & & \\ \hline
JavaScript & & & & \\ \hline
Haskell & & & & \\ \hline
Rust  & & & & \\ \hline
\end{tabular}
\end{center}

\end{exercise}

\end{document}
