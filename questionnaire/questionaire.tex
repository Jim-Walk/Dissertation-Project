\documentclass[11pt]{article} % Default font size is 12pt, it can be changed here

\usepackage{geometry} % Required to change the page size to A4
\usepackage{listings} % for code
\geometry{a4paper}
\usepackage[clear-aux]{xsim}

\usepackage{enumitem,amssymb,fmtcount}
\newlist{choices}{itemize}{1}
\setlist[choices]{label=$\Box$}
\newcommand*\choice{\item}

\DeclareExerciseProperty{choices}
\DeclareExerciseProperty*{multiple}
\DeclareExerciseEnvironmentTemplate{mc}
  {%
    \UseExerciseTemplate{begin}{default}%
    \IfExerciseBooleanPropertyTF{multiple}
      {}
      {%
        \GetExercisePropertyT{choices}
          {}%
      }%
    \begin{choices}
  }
  {%
    \end{choices}
    \UseExerciseTemplate{end}{default}%
  }

\DeclareExerciseType{mc}{
  exercise-env = multiplechoice ,
  solution-env = correctchoices ,
  exercise-name = Question ,
  solution-name = Solution ,
  exercise-template = mc ,
  solution-template = mc ,
  counter = exercise
}

\xsimsetup{
  exercise/name = Question ,
  solution/name = Solution
}

\lstset{basicstyle=\footnotesize}
\begin{document}
\author{Jim Walker}
\begin{center}
\large {High Performance Rust MSc Questionnaire}

\normalsize Jim Walker

s1893750@ed.ac.uk
\end{center}
\textbf{Questionnaire Information}
\newline\newline
\noindent\textbf{About this project:}
\newline
\noindent This project aims to evaluate the usability of Rust from the perspective of HPC programmers.

\noindent\textbf{Who is responsible for data collected?}

\noindent Jim Walker

\noindent\textbf{What is involved in this study?}

\noindent A multiple choice paper questionnaires which asks participants what particular fragments of Rust code do. Participants are also requested to self identify how proficient they are at the following programming languages: Fortran, C, C++, Python, Ruby, Java, JavaScript, Haskell and Rust. I will collect no other data from the participants.

\noindent "Responses will be digitised and used to create figures in my MSc dissertation. The data will be retained securely until the dissertation is marked, after which the data will be deleted. A secure back up will also be created and destroyed.

\noindent\textbf{What are the risks involved in this study?}

\noindent I do not anticipate any risks to participants. Exceptionally, people could try to ascertain which participants got higher marks on the questionnaire from the skill levels the participant applied to the various languages, but the risk of this affecting a participants future career progress would be negligible.

\noindent\textbf{What are the benefits of taking part in this study?}

\noindent People can test their knowledge on Rust. Once all data has been collected, correct answers will be circulated through the EPCC mailing list.

\noindent\textbf{What are your rights as a participant?}

\noindent Taking part in the study is voluntary. You may choose not to take part or subsequently cease participation at any time.

\noindent\textbf{Will I receive any payment or monetary benefits?}

\noindent No.

\noindent\textbf{For more information:}

\noindent You can contact Jim Walker directly, or his supervisor Magnus Morton, m.morton@epcc .ed.ac.uk
\pagebreak

\begin{multiplechoice}[choices=1]
What does the function \texttt{foo} do?
\begin{lstlisting}
fn foo(m: i32, n: i32) -> i32 {
   if m == 0 {
      n.abs()
   } else {
      foo(n % m, m)
   }
}
\end{lstlisting}
  \choice It finds the greatest common divisor of m and n
  \choice It doesn't compile.
  \choice It finds the closest prime number to n
  \choice It calls itself infinitely.
\end{multiplechoice}

\begin{multiplechoice}[choices=1]
In Rust, \texttt{vec!} is used to create a vector. All variables in Rust are immutable by default. What happens when we try to run this program?
\begin{lstlisting}
let v = vec![2,3];
v.push(3);
print!("{:?}", v);
\end{lstlisting}

\choice {[2,3,2]} is printed.
\choice {[2,2,2,3]} is printed.
\choice The program does not compile.
\choice The program compiles, but crashes when it tries to push 3 to v.

\end{multiplechoice}


\begin{multiplechoice}[choices=1]
Idomatic Rust code oten uses patterns associated with functional languages. Given an immutable vector, v, please select what the line of code below does.
\begin{lstlisting}
let a = v.iter().fold(1, |acc, x| acc * x);
\end{lstlisting}
  \choice Every element of v is set to 1, and then copied to a.
  \choice Every element of v is multiplied together and the result is stored in a.
  \choice Every element of v is multiplied by 1 and the result is stored in a.
  \choice The program does not compile.
\end{multiplechoice}

\begin{multiplechoice}[choices=1]
A vector's push method return an optional value, or none. What does this fragment of code print?
\begin{lstlisting}
let mut stack = Vec::new();

stack.push(1);
stack.push(2);
stack.push(3);

while let Some(top) = stack.pop() {
  print!("{} ", top);
}
\end{lstlisting}
  \choice Some(3) Some(2) Some(1)
  \choice 3 2 1 None None None...
  \choice 3 2 1
  \choice Some(3) Some(2) Some(1) None None None...
\end{multiplechoice}


\begin{multiplechoice}[choices=1]
What does this fragment of code do?

\begin{lstlisting}
let a: Vec<i32> = (1..).step_by(3)
                       .take(3)
                       .map(|x| x * 2)
                       .collect();
\end{lstlisting}

  \choice Sets a to {[2, 4, 6]}
  \choice The program doesn't compile.
  \choice {[4, 10, 16]}
  \choice {[2. 8, 14]}
\end{multiplechoice}

\begin{multiplechoice}[choices=1]
In this question, a and b are both vectors of the same length. The method \texttt{par\_chunks} returns a parallel iterator over at most \texttt{chunk\_size} elements at a time. What does this fragment of code do?
\begin{lstlisting}
a.par_chunks(chunk_size)
    .zip(b.par_chunks(chunk_size))
    .map(|(x,y)| x.iter()
                  .zip(y.iter())
                  .fold(0, | acc, ele| acc + *ele.0 * *ele.1)
    ).sum();

\end{lstlisting}

\choice Sum reduction
\choice Dot Product
\choice Element wise sum


\end{multiplechoice}
\begin{multiplechoice}[choices=1]
The Rust compiler's borrow checker makes sure that values are mutably borrowed if they are altered from a different function than the one they were created in. What does this program do?
\begin{lstlisting}
fn plus_one(x: &mut i32){
    *x += 1;
}
fn main(){
    let x = 64;
    plus_one(&mut x);
    println!("{}", x+1);
}
\end{lstlisting}
  \choice Print 65.
  \choice Prints an undefined value.
  \choice It doesn't compile.
  \choice Print 66.
\end{multiplechoice}
\pagebreak
\begin{exercise}
Please tick the boxes below to show your level of skill in the varying programming languages.

\begin{itemize}
  \item Basic knowledge: I am able to write loops, conditionals, and can name at least three data types in this language.
  \item {Comprehensive:} I can write large programs in this language. I am aware of the most common unique featues of the language, and understand some of them well enough for it to inform my programming in this language.
  \item {Advanced:} I have a deep understanding of the inner workings of this language. I can confidently and effectively use the more esoteric features of this language in my programs.

\end{itemize}

\begin{center}
\begin{tabular}{|l|c|c|c|c|}
\hline
& None & Basic & Comprehensive & Advanced \\ \hline
Fortran & & & & \\ \hline
C & & & & \\ \hline
C++ & & & & \\ \hline
Python & & & & \\ \hline
Ruby & & & & \\ \hline
Java & & & & \\ \hline
JavaScript & & & & \\ \hline
Haskell & & & & \\ \hline
Rust  & & & & \\ \hline
\end{tabular}
\end{center}

\end{exercise}

\end{document}
